%writeup for ENGR 202 Lab 4

\documentclass{article}
\usepackage{graphicx}
\graphicspath{ {images/} }
\title{Lab Write-up 4: Mystery Boxes}
\author{Simon Hannes, Zach Thompson, Kyle Peterson}
\begin {document}

\maketitle{}

\section*{Overview}
\paragraph{}
The purpose of this lab is to determine which components are within the Mystery Boxes. The first box examined contained one component while the second contained two components in series. 

\section*{Process}
\paragraph{}
To determine the content of the boxes, we connected a 1k resistor in series to the boxes, applied a DC voltage to the box and resistor, and measured the result. We noted the voltage response across the resistor and across the box. As expected, the sum of these two voltages was equal to the voltage applied. We then divided the voltage drop across the resistor by the known value of the resistor to determine current. All components are in series, and thus the current through the resistor is equal to the current through the box. Impedence of the box can then be determined by dividing voltage drop across the box by the current of the circuit. We then applied an AC current at five different frequencies and determined the phase difference of the resultant voltage drop vs the original applied voltage. We varied the frequency to produce an approximately $\angle{}45^\circ{}$ difference. 

\section*{Results}
\paragraph{}
Here are the numerical results we found for each respective box.

\subsection*{Box: Mercedes Benz}
\paragraph{}
We began by testing the resistor for accuracy using the multimeter and found it to be $979\Omega{}$. We also tested the box alone for resisitivity and found a very small value, suggesting an inductor. We connected a $5V$ DC power supply to the resistor and box in series, and set the multimeter to voltage mode. We found a $4.8V$ drop across the resistor and a $115mV$ drop across the box. Next, we connected a $5V$ AC source at $1KHz$ cooresponding to a source sinusoid of $5\cos{}(2000\pi{} t)$. The time delay between the source voltage and voltage measured across the box was found to be $190\mu{} s$.The phase shift is then given by: $\angle{} = 360^\circ{}(1000Hz)(190\mu{}) = 68.4^\circ{}$. 

\paragraph{} 
Next we found the voltage across the resistor by subtracting voltage drop across the box from the source voltage: $5\angle{}0^\circ{} - 0.28\angle{}68.4^\circ{} = 4.90\angle{}-3.04^\circ{}$ By dividing this result by the measured resistor value of $979\Omega$ we yield a series current of $0.005\angle{}-3.04^\circ{}A$. Overall impedance can now be found by dividing voltage drop across the box by the current through the series components: $Zb= -0.28\angle{}68.4^\circ{} / 0.005\angle{}-3.04^\circ{} = 55.90\angle{}71.44^\circ{}$. Since we're looking for the non-real component of impedence corresponding to what we believe to be an inductor, we take $55.90\sin{}71.44^\circ{} = 52.99 = wl$ dividing by $2000\pi{}$ yields $l=8.4mH$. We estimated that this was probably a 10mH inductor which was confirmed by the TA.   





